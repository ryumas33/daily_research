\documentclass[a4j]{jarticle}
\date{}
\usepackage{bm}
\usepackage{amsmath}
\usepackage{algorithm}
\usepackage[noend]{algpseudocode}
\algrenewcommand\algorithmicdo{}
\title{171010.1}

\begin{document}
\maketitle

\section{CATの項目選択基準として用いられる情報量(5)}
Wim J. van der Linden \& Cees A.W. Glas ``Computerized Adaptive Testing'' \cite{b1}のChapter 1の3. Modern Proceduresで紹介されている情報量についてまとめる(ほぼそのまま翻訳)。

\subsection{Fully Bayesian Criteria(3)}
事前事後分析に基づく最初の項目選択基準は最大期待情報基準である。この基準は$k$番目の項目に関する予測された応答について観測された情報量を最大化する。正式には、式(\ref{e1})のように表すことができる。

\newcommand{\argmax}{\mathop{\rm arg~max}\limits}
\begin{eqnarray}
\begin{split}
  \label{e1}
  i_{k} &\equiv \argmax_{j}\left\{p_{j}(0|u_{i_{1}},\ldots,u_{i_{k-1}})J_{u_{i_{1}},\ldots,u_{i_{k-1}},U_{j}=0}(\hat{\theta}_{u_{i_{1}},\ldots,u_{i_{k-1}},U_{j}=0})\right.\\
  &+\left.p_{j}(1|u_{i_{1}},\ldots,u_{i_{k-1}})J_{u_{i_{1}},\ldots,u_{i_{k-1}},U_{j}=1}(\hat{\theta}_{u_{i_{1}},\ldots,u_{i_{k-1}},U_{j}=1}):j\in R_{k}\right\}
\end{split}
\end{eqnarray}

式(\ref{e1})において観測された情報が事後分散によって置き換えられた場合、最小期待事後分散基準が得られる。

\newcommand{\argmin}{\mathop{\rm arg~min}\limits}
\begin{eqnarray}
\begin{split}
  \label{e2}
  i_{k} &\equiv \argmin_{j}\left\{p_{j}(0|u_{i_{1}},\ldots,u_{i_{k-1}}){\rm Var}(\theta|u_{i_{1}},\ldots,u_{i_{k-1}},U_{j}=0)\right.\\
  &+\left.p_{j}(1|u_{i_{1}},\ldots,u_{i_{k-1}}){\rm Var}(\theta|u_{i_{1}},\ldots,u_{i_{k-1}},U_{j}=1):j\in R_{k}\right\}
\end{split}
\end{eqnarray}

式(\ref{e2})は、推定器の二次の損失関数の下で事前事後リスクとなることが知られている。Owen(1975)はこの基準を\{資料の式(19)\}の彼の基準の数値的に複雑な代替案と呼んだ。

\begin{thebibliography}{9}
  \bibitem{b1} Wim J. van der Linden \& Cees A.W. Glas, ``Computerized Adaptive Testing'', pp.11-23, 2003, ISBN 0-7923-6425-2.
\end{thebibliography}

\end{document}