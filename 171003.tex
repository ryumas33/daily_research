\documentclass[a4j]{jarticle}
\date{}
\usepackage{bm}
\usepackage{amsmath}
\usepackage{algorithm}
\usepackage[noend]{algpseudocode}
\algrenewcommand\algorithmicdo{}
\title{171003}

\begin{document}
\maketitle

\section{CATの項目選択基準として用いられる情報量(3)}
Wim J. van der Linden \& Cees A.W. Glas ``Computerized Adaptive Testing'' \cite{b1}のChapter 1の3. Modern Proceduresで紹介されている情報量についてまとめる(ほぼそのまま翻訳)。

\subsection{Fully Bayesian Criteria(1)}
項目選択のためのすべてのベイジアン基準は$\theta$の事後分布に基づくなんらかの形の重み付けを含む。事後分布は尤度関数と事前分布の組み合わせであるため、ここまでの項目選択基準との基本的な相違は事前分布の仮定である。$\theta$の主観的な事前分布の選択方法の問題は扱わず、代わりに次のセクションでは被検者の背景変数(?)に関するデータから経験的な事前確率を推定する方法について説明する。このセクションの目的はvan der Linden(1998)で提案された項目選択のためのいくつかのベイジアン基準を再検討することのみである。
170928の式(1)と同様に、事後重み付け情報基準を定義することができる。

\newcommand{\argmax}{\mathop{\rm arg~max}\limits}
\begin{eqnarray}
  \label{e1}
  i_{k} \equiv \argmax_{j}\left\{\int I_{U_{j}}(\theta)g(\theta|u_{i_{1}},\ldots,u_{i_{k-1}})d\theta:j\in R_{k}\right\}
\end{eqnarray}

この基準は、事後分布の周り(?)で項目情報に重み付けをする。MAP推定値またはEAP推定値が使用される場合、基準はこれらの推定値において大きな情報量をもつ項目を優先する(?)。式(\ref{e1})の基準はフィッシャー情報量に基づくことに注意する。期待された情報と観測された情報の区別は3PLMに対してのみ実用的である(後で詳細を調べる)が、より多くのベイジアン選択は{フィッシャー情報量の期待値をとる前の式}で観測された情報を用いるであろう。また、式(\ref{e1})をカルバック・ライブラー情報量と組み合わせることも可能であることに注意する($I$を$K$にするのか)。

\begin{thebibliography}{9}
  \bibitem{b1} Wim J. van der Linden \& Cees A.W. Glas, ``Computerized Adaptive Testing'', pp.11-23, 2003, ISBN 0-7923-6425-2.
\end{thebibliography}

\end{document}