\documentclass[a4j]{jarticle}
\date{}
\usepackage{bm}
\usepackage{amsmath}
\usepackage{algorithm}
\usepackage[noend]{algpseudocode}
\algrenewcommand\algorithmicdo{}
\title{170928}

\begin{document}
\maketitle

\section{CATの項目選択基準として用いられる情報量(2)}
Wim J. van der Linden \& Cees A.W. Glas ``Computerized Adaptive Testing'' \cite{b1}のChapter 1の3. Modern Proceduresで紹介されている情報量についてまとめる(ほぼそのまま翻訳)。

\subsection{Likelihood-Weighted Information Criterion}
170927の式(4)のように未知パラメータ$\theta$について積分するのではなく、尤度をかけて可能なすべての$\theta$について積分することができる。この考え方は1997年にVeerkampとBergerによって提唱された。彼らはフィッシャー情報量のために提示したが、これは簡単にカルバック・ライブラー情報量に拡張することができる。
頻度論の枠組みでは、応答$U_{i_{1}}=u_{i_{1}},\ldots,U_{i_{k-1}}=u_{i_{k-1}}$に関連する尤度関数は、データが与えられたときの$\theta$の様々な尤度を示す。VerrkampとBergerは尤度関数を用いてフィッシャー情報量を重み付けし、式(\ref{e1})にしたがってk番目の項目を選択することを提案した。

\newcommand{\argmax}{\mathop{\rm arg~max}\limits}
\begin{eqnarray}
  \label{e1}
  i_{k} \equiv \argmax_{j}\left\{\int_{-\infty}^{\infty}L(\theta|u_{i_{1}},\ldots,u_{i_{k-1}})I_{i_{k}}(\theta)d\theta:j\in R_{k}\right\}
\end{eqnarray}

能力の最尤推定が行われる場合、式(\ref{e1})の基準は現在の推定値に近い$\theta$に大きな重みを置く。テストの開始期に尤度関数は平らであり、$\hat{\theta}_{k-1}$から離れた値は相当な重みを受ける。テストが終わりに向かうにしたがって尤度関数はより尖る傾向があり、すべての重みは$\hat{\theta}_{k-1}$に近い値にかかる。
VeerkampとBergerはまた、170927の式(4)のように現在の推定値に対する$\theta$の有限区間にわたる積分を仮定する区間情報量基準を示した。しかし、区間の大きさを新しいパラメータ$\delta_{k}$の関数として定義するのではなく、$\theta$の信頼区間を使用することを推奨している。170927の式(4)についても同様の示唆が与えられる。

\begin{thebibliography}{9}
  \bibitem{b1} Wim J. van der Linden \& Cees A.W. Glas, ``Computerized Adaptive Testing'', pp.11-23, 2003, ISBN 0-7923-6425-2.
\end{thebibliography}

\end{document}