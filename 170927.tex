\documentclass[a4j]{jarticle}
\date{}
\usepackage{bm}
\usepackage{amsmath}
\usepackage{algorithm}
\usepackage[noend]{algpseudocode}
\algrenewcommand\algorithmicdo{}
\title{170927}

\begin{document}
\maketitle

\section{CATの項目選択基準として用いられる情報量(1)}
Wim J. van der Linden \& Cees A.W. Glas ``Computerized Adaptive Testing'' \cite{b1}のChapter 1の3. Modern Proceduresで紹介されている情報量についてまとめる(ほぼそのまま翻訳)。

\subsection{予備知識}
適応テストにおいて初期の誤差の大きい推定能力値に対して項目選択を最適化してしまう問題は、テスト理論の減衰パラドックス(attenuation paradox)として知られる。
文中に登場する``calibration sample''および``capitalization on chance''については別の機会に調べる。

\subsection{Maximum Global-Information Criterion}
試験開始時の大きな推定誤差に対処するため、ChangとYingはフィッシャー情報量の代わりに2つの確率分布の差異を測る尺度であるカルバック・ライブラー情報量を用いることを提案した。
未知の真値$\theta_{0}$の尤度関数と現在の推定値$\hat{\theta}_{k-1}$の尤度関数に対するカルバック・ライブラー情報量は式(\ref{e1})となる。

\begin{eqnarray}
  \label{e1}
  K_{i_{k}}(\hat{\theta}_{k-1},\theta_{0}) \equiv E[\log\frac{L(\theta_{0}|U_{i_{k}})}{L(\hat{\theta}_{k-1}|U_{i_{k}})}]
\end{eqnarray}

式(\ref{e1})の期待値は応答変数$U_{ik}$によるため、式(\ref{e2})と計算できる。

\begin{eqnarray}
  \label{e2}
  K_{i_{k}}(\hat{\theta}_{k-1},\theta_{0}) = p_{i_{k}}(\theta_{0})\log\frac{p_{i_{k}}(\theta_{0})}{p_{i_{k}}(\hat{\theta}_{k-1})} + [1-p_{i_{k}}(\theta_{0})]\log\frac{1-p_{i_{k}}(\theta_{0})}{1-p_{i_{k}}(\hat{\theta}_{k-1})}
\end{eqnarray}

応答変数間には条件付き独立性があるため、テストの最初の$k$個の項目の情報量は式(\ref{e3})のように書くことができる。

\begin{eqnarray}
  \label{e3}
  K_{k}(\hat{\theta}_{k-1},\theta_{0}) \equiv E[\log\frac{L(\theta_{0}|U_{i_{1}},\ldots,U_{i_{k}})}{L(\hat{\theta}_{k-1}|U_{i_{1}},\ldots,U_{i_{k}})}] = \sum_{h=1}^{k}K_{i_{h}}(\hat{\theta}_{k-1},\theta_{0})
\end{eqnarray}

カルバック・ライブラー情報量は、応答変数が$\hat{\theta}_{k-1}$と$\theta_{0}$とをいかにうまく区別しているかを示す。$\theta_{0}$は未知であるため、ChangとYingは式(\ref{e1})を式(\ref{e4})のように現在の推定値についての区間の積分で置き換えることを提案している。

\newcommand{\argmax}{\mathop{\rm arg~max}\limits}
\begin{eqnarray}
  \label{e4}
  i_{k} \equiv \argmax_{j}\left\{\int_{\hat{\theta}_{k-1}-\delta_{k}}^{\hat{\theta}_{k-1}+\delta_{k}}K_{j}(\hat{\theta}_{k-1},\theta)d\theta:j\in R_{k}\right\}
\end{eqnarray}

\begin{thebibliography}{9}
  \bibitem{b1} Wim J. van der Linden \& Cees A.W. Glas, ``Computerized Adaptive Testing'', pp.11-23, 2003, ISBN 0-7923-6425-2.
\end{thebibliography}

\end{document}